% !TEX TS-program = pdflatex
% !TEX encoding = UTF-8 Unicode

% This is a simple template for a LaTeX document using the "article" class.
% See "book", "report", "letter" for other types of document.

\documentclass[11pt]{article} % use larger type; default would be 10pt

\usepackage[utf8]{inputenc} % set input encoding (not needed with XeLaTeX)

%%% Examples of Article customizations
% These packages are optional, depending whether you want the features they provide.
% See the LaTeX Companion or other references for full information.

%%% PAGE DIMENSIONS
\usepackage{geometry} % to change the page dimensions
\geometry{a4paper} % or letterpaper (US) or a5paper or....
% \geometry{margin=2in} % for example, change the margins to 2 inches all round
% \geometry{landscape} % set up the page for landscape
%   read geometry.pdf for detailed page layout information

\usepackage{graphicx} % support the \includegraphics command and options

% \usepackage[parfill]{parskip} % Activate to begin paragraphs with an empty line rather than an indent

%%% PACKAGES
\usepackage{booktabs} % for much better looking tables
\usepackage{array} % for better arrays (eg matrices) in maths
\usepackage{paralist} % very flexible & customisable lists (eg. enumerate/itemize, etc.)
\usepackage{verbatim} % adds environment for commenting out blocks of text & for better verbatim
\usepackage{subfig} % make it possible to include more than one captioned figure/table in a single float
% These packages are all incorporated in the memoir class to one degree or another...

%%% HEADERS & FOOTERS
\usepackage{fancyhdr} % This should be set AFTER setting up the page geometry
\pagestyle{fancy} % options: empty , plain , fancy
\renewcommand{\headrulewidth}{0pt} % customise the layout...
\lhead{}\chead{}\rhead{}
\lfoot{}\cfoot{\thepage}\rfoot{}

%%% SECTION TITLE APPEARANCE
\usepackage{sectsty}
\allsectionsfont{\sffamily\mdseries\upshape} % (See the fntguide.pdf for font help)
% (This matches ConTeXt defaults)

%%% ToC (table of contents) APPEARANCE
\usepackage[nottoc,notlof,notlot]{tocbibind} % Put the bibliography in the ToC
\usepackage[titles,subfigure]{tocloft} % Alter the style of the Table of Contents
\renewcommand{\cftsecfont}{\rmfamily\mdseries\upshape}
\renewcommand{\cftsecpagefont}{\rmfamily\mdseries\upshape} % No bold!


%%% Source Code Appearance
\usepackage{listings}
\usepackage[usenames,dvipsnames]{color}

\definecolor{gray}{rgb}{0.4,0.4,0.4}
\definecolor{darkblue}{rgb}{0.0,0.0,0.6}
\definecolor{cyan}{rgb}{0.0,0.6,0.6}

\lstset{
  basicstyle=\ttfamily,
  columns=fullflexible,
  showstringspaces=false,
  commentstyle=\color{gray}\upshape
  breaklines=true
}

\lstdefinelanguage{XML_new}
{
  morestring=[b]",
  morestring=[s]{>}{<},
  morecomment=[s]{<?}{?>},
  stringstyle=\color{black},
  identifierstyle=\color{darkblue},
  keywordstyle=\color{cyan},
  morekeywords={seq,drid,confcnt, totcnt, spid, flid,cov, type,url, cid, drconf}% list your attributes here
}

\lstdefinelanguage{DTD}
{
morestring=[b]",
  morecomment=[s]{<?}{?>},
%morecomment=[s]{\#}{\>},
%moredirectives={\#REQUIRED},
  stringstyle=\color{darkblue},
  identifierstyle=\color{gray},
  keywordstyle=\color{cyan},
  emphstyle=\color{RoyalPurple},
emph={CDATA,PCDATA,REQUIRED, IMPLIED},
 moredelim=*[s][\color{Black}]{(}{)},
morekeywords={ELEMENT, ATTLIST, DOCTYPE,  EMPTY}% list your attributes here
}

%%% END Article customizations

%%% The "real" document content comes below...

\title{The CRISPR File Specification (v1.0)}
\author{Michael Imelfort and Connor Skennerton}
\date{18th November 2011} % Activate to display a given date or no date (if empty),
         % otherwise the current date is printed 

\begin{document}
\maketitle

\section{Introduction}

The CRISPR file (.crispr) is an xml based document designed to describe all properties of the direct repeats and spacer sequences obtained from CRISPR loci from genomic and metagenomic datasets.  It's key function is to enable a standard way of describing the complex arrangements of spacers that are seen in environmental samples.  Each .crispr file can contain multiple  \lstinline[language=XML_new]$<group>$   tags; each one corresponding to a different CRISPR locus.  Each group then has three tags: \lstinline[language=XML_new]$<data>$, \lstinline[language=XML_new]$<metadata>$ and \lstinline[language=XML_new]$<assembly>$ 
\subsection{Terminologies and Concepts}
\subsubsection{Clustered Regularly Interspersed Short Palindromic Repeats (CRISPRs)}
CRISPRs are a class of repeatative elements found in many bacterial and archaeal genomes.  They encode a complex microbial immune system capable of rapidly adapting to, and targeting invading DNA.  CRISPR elements are composed of direct repeats (DRs), spacers and flanking sequences.
\paragraph{Direct Repeats}
Although the P in CRISPR is for palindromic, most DRs do not form perfect palindromes.  There are many different types of repeats, some that do contain palindromic regions in them and others that do not.  What is common is that within a CRISPR locus the sequence of the DR will remain identical. This is essential as the DR sequence is used to bound the spacer sequences that are used for the immune response.
\paragraph{Spacers}
Spacer sequences are small segments of invading DNA that is added to a CRISPR locus.  These spacers are used as a targeting mechanism of degradative proteins. 
\paragraph{The Leader Sequence}
The leader seuqence flanks one side of the the CRISPR locus and is thought to act as a promoter sequence.
\section{Elements}
\subsection{Data}
The data tag is designed list all of the spacers that a CRISPR contains as well as the direct repeats and any flanking sequences such as the leader sequence.  The plural is intended for the direct repeats as a single locus could have multiple variants of the direct repeat.

\begin{lstlisting}[language=XML_new]
<data>
	<drs>
		<dr seq="GAGTCCCCGC" drid="DR1" confcnt="12" totcnt="20"/>
	</drs>
	<spacers>
		<spacer seq="TGAGCGGTCGC" spid="SP1"/>
	</spacers>
	<flankers>
		<flanker seq="GGAGTTCTAC" flid="FL1"/>
	</flankers>
</data>
\end{lstlisting}
\subsubsection{Direct Repeats}
Direct repeats are defined in the \lstinline[language=XML_new]$<drs>$  tag as a \lstinline[language=XML_new]$<dr>$.  Each \lstinline[language=XML_new]$<dr>$  tag has two requiredattributes: \lstinline[language=XML_new]$seq$ and \lstinline[language=XML_new]$drid$; and two implied attributes: \lstinline[language=XML_new]$confcnt$ and \lstinline[language=XML_new]$totcnt$

\begin{lstlisting}[language=XML_new]
<drs>
	<dr seq="GAGTTCCCCGCGCCA" drid="DR1" confcnt="12" totcnt="20"/>
</drs>
\end{lstlisting}

The \lstinline[language=XML_new]$seq$ attribute is the sequence of the direct repeat in its lowest lexicographical form and should not contain any characters other than the IUPAC standard for DNA bases and should be in uppercase.   The \lstinline[language=XML_new]$drid$ attribute is set by the programer, it must be an integer prefixed with "DR" however it does not need to be sequential -- just unique for the \lstinline[language=XML_new]$<group>$ that it is contained in.  

The two implied attributes \lstinline[language=XML_new]$confcnt$ and \lstinline[language=XML_new]$totcnt$

\subsubsection{Spacers}
Spacers are defined inside the \lstinline[language=XML_new]$<spacers>$ tag as an empty element containing two required attributes \lstinline[language=XML_new]$seq$ and \lstinline[language=XML_new]$spid$, as well as an implied attribute \lstinline[language=XML_new]$cov$.
\begin{lstlisting}[language=XML_new]
<spacers>
	<spacer seq="TGAGCGGTCGC" spid="SP1" cov="10"/>
</spacers>
\end{lstlisting}
The \lstinline[language=XML_new]$seq$ attribute is listed in the same directionality as the direct repeat.  This is not necessarily is the lowest lexicographical form for the spacer but will be the form in which the direct repeat is in it's lowest lexicographical form.  The \lstinline[language=XML_new]$spid$ attribute is set by the programer, it must be an integer prefixed with "SP" however it does not need to be sequential -- just unique for the \lstinline[language=XML_new]$<group>$ that it is contained in.  The \lstinline[language=XML_new]$cov$ attribute describes how many instances of this spacer were seen in the genome/metagenome/dataset, which may be used in assembly to resolve forks in the arrangement of spacers or to determine community abundance of particular spacers. 
\subsubsection{Flankers}
The \lstinline[language=XML_new]$<flankers>$ element is not a required part of \lstinline[language=XML_new]$<data>$ however if present decribes any flanking sequences found on either side of the spacer array, such as the leader sequence. Each \lstinline[language=XML_new]$<flanker>$ contains two required attributes: \lstinline[language=XML_new]$seq$ and \lstinline[language=XML_new]$flid$.
\begin{lstlisting}[language=XML_new]
<flankers>
	<flanker seq="GGAGTTCTAC" flid="FL1"/>
</flankers>
\end{lstlisting}
The requirements for these attributes are the same as for spacers, namely that the sequence be in the form to have the associated direct repeat be in its lowest lexicographical form; and the \lstinline[language=XML_new]$flid$ being an integer, unique to a \lstinline[language=XML_new]$<group>$.
\subsection{Metadata}
The \lstinline[language=XML_new]$<metadata>$ tag can hold a range of information that links with a particular group.  The metadata can take two forms, as either a \lstinline[language=XML_new]$<notes>$ tag that holds freeform infromation inputed by the user or outputed by a program; or as a URL to a local or remote file.
\begin{lstlisting}[language=XML_new]
<metadata>
	<notes>
		E. coli dataset of ca. 20Million 100bp reads
		Command run: crass -o test_out/ -l 4 ./test_input
	</notes>
	<file type="image" url="./crispr1.jpg"/>
	<file type="sequence" url="./crispr1.fa"/>
	<file type="data" url="./crispr1_mimmarks.gcdml"/>
</metadata>
\end{lstlisting}

\subsubsection{Notes}
The \lstinline[language=XML_new]$<notes>$ tag should not be a dump for large amounts of internal or external data (such as a program log) but instead be a place for the user to store information about program parameters or basic dataset information.  More extensive data and information should be referred to using the \lstinline[language=XML_new]$<file>$ tag (see below).

\subsubsection{External Files}
The \lstinline[language=XML_new]$<file>$ tag can be used to refer to a file that contains extra data about this CRISPR pointed to by the \lstinline[language=XML_new]$url$ attribute.  There are currently three acceptable types of files defined by the \lstinline[language=XML_new]$type$ attribute: image, sequence and data.  Image files could show a graphical representation of the spacer arrangment produced by programs such as Graphviz.  Sequence files refer to fasta or fastq files containing reads that belong to this CRISPR. The final option is generic and may reference any file that holds extra information about the dataset, such as a MIGS/MIMS standards compliant metadata file. 
\subsection{Assembly}
The \lstinline[language=XML_new]$<assembly>$ tag defines how all of the spacers, flankers and direct repeats defined in the \lstinline[language=XML_new]$<data>$ section are arranged together.  An \lstinline[language=XML_new]$<assembly>$ tag contains a one or more \lstinline[language=XML_new]$<contig>$ tags that contain references to individual spacers.  
\begin{lstlisting}[language=XML_new]
<assembly>
	<contig cid="C2">
		<cspacer spid="SP6">
			<bspacers>
				<bs spid="SP36" drid="DR1" drconf="0" />
			</bspacers>
			<fspacers>
				<fs spid="SP27" drid="DR1" drconf="0" />
			</fspacers>
		</cspacer>
		<concensus>
			TCAGCTTTATAAATCCGGAGATACGGAAACTA
		</concensus>
	</contig>
</assembly>
\end{lstlisting}
\subsubsection{Contigs}
Contigs are linear sections of a graph where there are no branches between spacers, that is to say that every spacer in a contig contains a single forward and backward link.  Therefore spacers which have multiple forward or backward spacers links are placed into a contig on their own.  This is to make it easy to assembly and join multiple different branching paths through a CRISPR locus as spacers that join multiple divergent arrangments would be in both pathways.

Each contig is defined with a \lstinline[language=XML_new]$<contig>$ tag, which requires a  \lstinline[language=XML_new]$cid$ attribute that must be unique to the \lstinline[language=XML_new]$<group>$.  A \lstinline[language=XML_new]$cid$ must be an integer prefixed with "C".  Each \lstinline[language=XML_new]$<contig>$ must contain one or more \lstinline[language=XML_new]$<cspacer>$ tags and may also contain a \lstinline[language=XML_new]$<concensus>$ tag.
\paragraph{\lstinline[language=XML_new]$<cspacer>$}
contains a single required attribute \lstinline[language=XML_new]$spid$ that refers to one of the spacers defined in the \lstinline[language=XML_new]$<data>$ tag.  It should also contain references to any linked \lstinline[language=XML_new]$<spacer>$ or \lstinline[language=XML_new]$<flanker>$ sequences using the \lstinline[language=XML_new]$<bspacers>$, \lstinline[language=XML_new]$<fspacers>$, \lstinline[language=XML_new]$<bflankers>$ and \lstinline[language=XML_new]$<fflankers>$ tags, although if a spacer does not connect to anything, then none of these tags will be present.
\subparagraph{\lstinline[language=XML_new]$<bspacers>$ \& \lstinline[language=XML_new]$<fspacers>$}
These two tags contain lists of the forward linking and backward linking spacers for the current  \lstinline[language=XML_new]$<cspacer>$.  There does not ned to be both types represented for a \lstinline[language=XML_new]$<cspacer>$ -- there may be neither if the spacer joins only to flankers.  The \lstinline[language=XML_new]$<fs>$ and \lstinline[language=XML_new]$<bs>$ tags denote the forward and backward spacers, respectively.  They have a single required attribute \lstinline[language=XML_new]$spid$ as well as two implied attributes \lstinline[language=XML_new]$drid$ and \lstinline[language=XML_new]$drconf$.  If no \lstinline[language=XML_new]$drid$ attribute is provided then the default direct repeat for the group will be used.
\subparagraph{\lstinline[language=XML_new]$<bflankers>$ \& \lstinline[language=XML_new]$<fflankers>$}
These two tags work in the same way as \lstinline[language=XML_new]$<bspacers>$ and \lstinline[language=XML_new]$<fspacers>$ and contain information about any flanking sequences that join to this spacer.  Each individual flanker is marked with either \lstinline[language=XML_new]$<bf>$ or \lstinline[language=XML_new]$<ff>$ tags.  
\paragraph{Concensus}

\section{Document Type Definition (DTD)}
\lstinputlisting[language=DTD]{../resources/crass.dtd}

\section{Example}
\end{document}
